\input style.tex

\sec Úvod

Program je implementován v~programovacím jazyce C.
Měření bylo provedeno na školním počítači s~procesorem Intel(R) Core(TM) i7-6700 CPU @ 3.40GHz, 16 GB paměti RAM a~operačním systémem Gentoo Linux.
Program je přeložen kompilátorem clang verze 7.0.0 s~přepínači {\tt -std=c99 -O2}.

Implementace se v~některých místech liší od přednášené verze.
Operace {\algo insert} rovnou vkládá přidávaný prvek do haldy, nezdržuje se tvorbou jednoprvkové haldy a~{\algo mergováním}.
Při {\algo extract minu} se potomci odebíraného vrcholu připojí do seznamu haldových stromů najednou jako sjednocení dvou spojových seznamů místo postupného vkládání.
V~konsolidaci se velikost pole pro jednotlivé řády haldových stromů neurčuje z~počtu prvků haldy, ale spočítá se jako {\tt maximální řád haldových stromů + počet stromů + 1}.
Informace o maximálním řádu haldových stromů se průběžně počítá po operacích {\algo extract minu} a {\algo decrease key}.
Operací {\algo FHCut} se v~implementaci myslí něco jiného než popis v~\cite[pruvodce], jde přímo o odtržení syna od rodiče.

Pokud se v~operaci {\algo decrease key} priorita klíče nesnížila (tzn. \uv{snižuje} se na hodnotu, která je stále větší než původní), pak se tato operace nepočítá jako provedený {\algo decrease key}.
To může ovlivnit některé výsledky měření.

\sec Rovnoměrný a nevyvážený test

Rovnoměrný test provádí rovnoměrně mnoho operací {\algo insert}, {\algo extract min} a {\algo decrease}, zatímco nevyvážený test při stejném počtu operací provádí řádově více operací~{\algo decrease}.

Měření standardní Fibonacciho haldy je v grafu~\ref[standard-avg-steps].
Pro přehlednost je graf rozdělen na dva grafy~\ref[standard-uniform-avg-steps] a \ref[standard-imbalanced-avg-steps].
V~obou testech složitost stoupá logaritmicky, což odpovídá tomu, že amortizovaná složitost operace {\algo extract min} je logaritmická.
Oscilaci grafu nejspíše způsobil generátor.
Výsledek úplně neodpovídá očekávání -- na první pohled bychom si řekli, že pokud je provedeno \uv{více konstantních operací} na úkor logaritmických operací, tak by počet kroků měl být menší.
Počet kroků v operaci {\algo extract min} je ovlivněn počtem synů extrahovaného vrcholu a počet stromů v haldě.
Nezdá se, že by generátor cíleně extrahoval vrcholy s velkým počtem synů, proto se zaměřme na počet stromů v haldě.
Většina operací {\algo decrease} (z~pohledu statistiky, ne generátoru) bude mířit na vrcholy, které nejsou v~kořenech, a přidají strom do seznamu haldových stromů.
Jelikož se nevyvážený test od rovnoměrného testu liší v počtu operací {\algo decrease}, pak rozdíl můžeme odůvodnit tím, že ačkoliv počty kroků strávené odtrháváním synů od extrahovaného vrcholu budou v obou testech srovnatelné, počet haldových stromů v konsolidaci bude řádově více.

\inspicwcap{outputs/graph-1.pdf}{standard-avg-steps}{Průměrný počet kroků operace {\algo extract min} standardní Fib. haldy.}

\inspicwcap{outputs/standard-uniform-avg-steps.pdf}{standard-uniform-avg-steps}{Průměrný počet kroků operace {\algo extract min} standardní Fib. haldy v~uniformním testu.}

\inspicwcap{outputs/standard-imbalanced-avg-steps.pdf}{standard-imbalanced-avg-steps}{Průměrný počet kroků operace {\algo extract min} standardní Fib. haldy v~nevyváženém testu.}

\sec Zákeřný test

Měření zákeřného testu je v grafu~\ref[cunning].
V~tomto testu se zdá, že obě haldy fungují téměř stejně, což je špatně, příčinu chyby se však nepodařilo najít.
To však nebrání tomu, abychom si popsali, co se v~zákeřném testu děje.

V~zákeřném testu generátor vytváří haldu na $n$ prvcích, kde soubor stromů obsahuje stromy izomorfní hvězdám~$S_0, S_1, \ldots, S_{\Theta(\sqrt{n})}$, což lze vyčíst z článku~\cite[fib], který je přiložený v generátoru.
To je možné jenom díky tomu, že naivní verze Fibonacciho haldy dovolí, aby stromy měly takový tvar, z~přednášky víme, že v korektním stromě má $i$-tý syn alespoň $i-2$ synů, což se v~naivní variantě nemusí dít.
Tvorba hvězdy~$S_n$ ani není (myšlenkově) náročná, stačí vytvořit dvě hvězdy~$S_{n-1}$, konsolidací donutit, aby se {\algo zmergovaly} a následně vyškubat všechno, co přebývá.
V této haldě pak operace {\algo extract min} trvá $\Omega(\sqrt{n})$, kde $n$ je počet prvků haldy.
To je vynuceno tím, že se přidají dva prvky~$v_1$ a $v_2$, kde $p(v_1)$ je minimum a $p(v_2)$ je druhá nejmenší priorita, $p$ označuje prioritu vrcholu.
Poté se oba extrahují, po první extrakci se pod vrchol $v_2$ napojí všechny stromy a po druhé extrakci máme opět soubor hvězd.
To v~naivní Fibonacciho haldě zjevně vynutí $\Omega(\sqrt{n})$ kroků.
Podotkněme, že cyklus dvou {\algo insertů} a {\algo extract minů} je proveden exponenciálně-krát, aby se tento efekt do grafu projevil.

\inspicwcap{outputs/graph-2.pdf}{cunning}{Průměrný počet kroků operace {\algo extract min} v~zákeřném testu. {\bf Tento graf není správný.} Příčinu chyby se nepodařilo najít.}

\sec Hluboký test

Výsledky měření tohoto testu jsou v grafech~\ref[deep-extract] a \ref[deep-decrease].
V~hlubokém testu generátor vytváří haldu o $n$ prvcích, kde jeden strom je izomorfní cestě na $\Theta(n)$ vrcholech a poté provede {\algo decrease} na list cesty.
Na cvičeních jsme si ukazovali, že tvorba cesty způsobí, že všechny vrcholy na této cestě budou označené.
Provedení operace {\algo decrease} ve standardní variantě na list této cesty způsobí {\algo FHCut} na každý vrchol této cesty, zatímco v naivní variantě se odtrhne pouze list.
To také odpovídá tomu, že maximální počet kroků roste lineárně a vysvětluje, proč v~tomto testu má naivní varianta lepší operaci {\algo decrease}.
Maximální počet kroků operace {\algo extract min} také stoupá lineárně, což je nejspíše způsobeno konsolidací jednoprvkových vrcholů, které vznikly po provedení {\algo FHCut} na všech vrcholech cesty a jejich přesunem do seznamu stromů.
Průměry jsou si však velmi podobné, tento jev je tedy spíše anomálií než pravidlo.

\inspicwcap{outputs/graph-3.pdf}{deep-extract}{Průměrný počet kroků operace {\algo extract min} v~hlubokém testu.}

\inspicwcap{outputs/graph-4.pdf}{deep-decrease}{Průměrný počet kroků operace {\algo decrease} v~hlubokém testu.}

\sec Literatura

\usebibtex{bib}{alpha}

\bye

